\begin{frame}
	\frametitle{Local elections 2014 - results summary}
	\begin{itemize}
	\item Marienbad - 5 mandates, majority with 21\%
	\item Prague 4 Pirates + 7 Pirates \& 5 supporters in districts
	\item Brno
	\item Majet\'in 1 Pirate
	\item Prost\v{e}jov 1 Pirate (coalition)
	\item \v{S}umperk 1 independent (coallition)
%%	\item Olomouc (coallition) - Pirate didn't get a seat, possi
%%	\item P\v{r}erov (coallition) - 
%%	\item Jesen\'ik (pseudocoalition - Pirates under greens)
	\item St\v{r}\'ibro 1 Pirate
	\item Liberec 2 independent
	\end{itemize}
\end{frame}
\begin{frame}
	\frametitle{Marienbad}
	\begin{itemize}
	\item 5 mandates, majority with 21\%
	\item vast policy, clearly commenting on local issues
	\item trustworthy candidates with long-term activity record
	\item PR newspaper distributed to every mailbox in the city
	\item main policies presented in articles
	\item posters with slogans
	\item website pirati.ml, social network activity
	\item local VyOseni, underground art community
	\item incapability of other politicians in the city
	\end{itemize}
\end{frame}
\begin{frame}
	\frametitle{Prague}
	
\end{frame}
\begin{frame}
	\frametitle{Brno}
\end{frame}
\begin{frame}
	\frametitle{Notes \& experience}
	\begin{itemize}
		\item election system allows "preferential votes"
		\item often happened that in coallitions Pirate candidates were "outvoted" on lower positions
		\item better poll attendance of coallition partners voters etc.
	\end{itemize}
\end{frame}
\begin{frame}
	\frametitle{General policy for communal elections}
\end{frame}
\begin{frame}
	\frametitle{After elections - presidium's general strategy}
	\begin{itemize}
		\item Piratecon - improve qualification
		\item how to be a good representative
		\item how to introduce free software and OpenData
		\item how to assign work to officers
		\item YOU'RE WELCOME TO VISIT
	\end{itemize}
\end{frame}
\begin{frame}
	\frametitle{Summary - how to score}
\end{frame}